\documentclass[a4paper,11pt]{article}
\usepackage[T1]{fontenc}
\usepackage[utf8]{inputenc}
\usepackage{lmodern}
\usepackage[italian]{babel}
\usepackage{graphicx}
\begin{document}
\title{Note per briscola}
\author{Marco Marini}
\maketitle
\tableofcontents
\part{Introduzione}
Note per l'implementazione dell'algoritmo di strategia di gioco della briscola.

\section{Strategia}

Per decidere quale strategia applicare si applica la teoria dei giochi.

Viene definito lo stato corrente \marginpar{Stato corrente} o stato reale di gioco da parte del
giocatore IA con gli elementi d'informazioni necessari:
le carte possedute dal giocatore, la briscola se non è stata già pescata, le carte dell'avversario nel
caso si sia in finale di partita,
i punti guadagnati dal giocatore e quelli guadagnati dall'avversario, la carte giocata dall'avversario
se era il suo turno e le carte ancora nel mazzo se non si è in finale di partita.

\subsection{Turno IA}
Analizziamo ora il caso che sia il giocatore IA ad avere la prima giocata. 

Partendo dallo stato corrente si valuta la probabilità di vincita migliore alla fine della mano corrente
\marginpar{Fine mano}, ovvero quando l'avversario ha giocato la propria carta.
Per sceglire la strategia di gioco vengono valutati tutti i possibili stati di fine mano per ogni carta
posseduta dal giocatore e per ogni miglior giocata tra le possibile combinazione di carte che
l'avverario putrebbe possedere.
La miglior giocata per l'avversario è quella che produce la maggior probabilità di vincita per lui.

Le probabilità di vincita del giocatore e dell'avversario non sono complementari
\marginpar{Non complementarietà} in quando in caso di pareggio sono entrambe nulle
è quindi necessario valutare separatamente le due probabilità.

Lo stato di fine mano è simile ad uno stato corrente \marginpar{Stato virtuale} dove in aggiunta
si conoscono anche  le carte potenzialmente in mano all'avversario.
La valutazione di uno stato di fine mano avviene in modo ricorsivo (deep search) analizzando
l'albero di evoluzione del gioco 
fino ad arrivare all'ultima giocata oppure limitando la ricerca fino ad un determinato livello
di profondità e applicando
in tal caso regole empiriche di valutazione meno precise.

Oltre allo stato di fine mano è utile considerare gli stati intermedio di gioco cioè quando
il giocatore che ha il turno di giocata
ha scartato oppure lo stati virtuale di inizio mano.

Vediamo il caso di stato virtuale di inizio mano \marginpar{Inizio mano} con turno IA.
Questo caso si ha quando da uno stato di fine mano vinto da IA si considera una combinazione
di carte pescate dal mazzo.
Quindi se $ n $ è il numero di carte ancora nel mazzo, si avrano $ n (n - 1) $ combinazioni
possibili di stati virtuali di inizio mano.

Nel caso ci sia una sola carte nel mazzo $ n = 1 $, lo stato sucessivo sara quello di finale
\marginpar{Finale} di partita in quanto IA pesca la carta nascosta del mazzo e l'avversario pesca
la briscola e tutte le carte a quel punto sono conosciute da ambedue i giocatori.


\subsection{Turno avversario}

Nel caso l'avversario abbia la prima giocata lo stato corrente contiene anche la carta giocata
dall'avversario e quindi la strategia di gioco sarà quella di valutare la miglior giocata tra le carte
disponibile al giocatore generando quindi tutti gli stati virtuali di fine mano e sciegliendo quello
con probabilità di vincita più alta.

Vediamo il caso di stato virtuale di inizio mano \marginpar{Inizio mano} con turno avversario.
Questo caso si ha quando da uno stato di fine mano vinto da avversario si considera
una combinazione di carte pescate dal mazzo.
Quindi se $ n $ è il numero di carte ancora nel mazzo, si avrano $ n (n - 1) $ combinazioni
possibili di stati virtuali di inizio mano.

La valutazione procede prendendo in considerazione la probabilità di vincita del giocatore per
la miglior carte giocata dall'avversario la miglior carta del giocatore.

\subsection{Transizione di stato}

Partendo dagli stati correnti di gioco possono essere generati vari tipi di stati virtuali.


Nel caso di ultima mano \marginpar{Ultima mano} non è necessaria alcuna strategia in quanto
 non ci sono scelte da fare.

Nel caso di finale \marginpar{Finale} di gioco si identificano due tipi di stato in base al numero
di carte da giocare: penultima mano e terzultima mano.
Per ogniuna dei due stati possiamo avere il caso di turno avversario e il caso di turno IA
quindi avremo:
\begin{itemize}
\item
penultima mano con turno avversario dove il giocatore ha quindi solo due carte da giocare
che generano quindi due stati di ultima mano,
\item
penultima mano turno IA dove il giocatore ha 2 carte da giocare e altrettane possibilità l'avversario.
Nel qual caso devono essere generati 2 stati virtuali intermedi e per ogniuno devono essere generati
i 2 fine mano con le risposte avversario
\item
terzultima mano turno avversario dove il giocatore ha 3 carte da giocare che generano 3 stati
virtuali di finale,
\item
terzultima mano turno IA dove il giocatore ha 3 carte da giocare e altreattante l'aversario.
Vengono generati 3 stati intermedi e per ogniuno sono generati 3 stati di fine mano con
le risposte avversarie
\end{itemize}

Nello stato corrente \marginpar{Stato corrente} durante il gioco abbiamo le due casistiche
in base al turno.
Nel caso del turno IA avremo tre stati intermedi per ogni possibile giocata. per ogniuno
di questi dovranno essere generate tutte le combinazioni di carte che l'avversario può avere.
Se $ N $ è il numero di carte nel mazzo ci saranno 
\[
\left( \stackrel{N}{3} \right)  = \frac{N!}{3! (N - 3)! } = \frac{N (N - 1) (N - 2)}{6}
\]
Per ogniuno di questi poi si dovranno considerare i 3 stati di fine mano prodotti da ogni carta
giocata.

Nel caso di turno avversario il procedimento è invertito per cui si considerano tutte gli stati
intermedi generati dalle combinazioni di 3 carte in mano avversaria tra le N presenti nel mazzo,
per ognuno le tre possibili scelte dell'avversario (stato intermedio) e poi i tre stati di fine di
mano derivati dalla scelte scelte possibili del giocatore.

Prendiamo poi in considerazione le transizioni dagli stati di fine mano \marginpar{Fine mano}
agli stati di inizio mano.
Si prendono quindi in considerazione le $ N (N-1) $ combinazioni di pescate dei 2 giocator
per riformare le carte in mano e il turno sucessivo viene determinato dal vincitore della mano.

\end{document}